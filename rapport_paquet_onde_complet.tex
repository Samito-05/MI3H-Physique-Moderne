
\documentclass[11pt]{article}
\usepackage[utf8]{inputenc}
\usepackage{amsmath,amssymb,graphicx,geometry,hyperref}
\geometry{margin=2.5cm}
\title{Simulation numérique de la propagation d'un paquet d'onde dans un puits de potentiel fini}
\date{\today}
\begin{document}
\maketitle

\section*{Objectif}
Ce document présente une simulation numérique de la propagation d’un paquet d’onde gaussien dans un potentiel fini (puits ou barrière), ainsi que le calcul des états stationnaires dans ce potentiel.

\section*{Modèle physique}
On considère une particule quantique décrite par l'équation de Schrödinger dépendante du temps :
\[
i \hbar \frac{\partial \psi}{\partial t} = -\frac{\hbar^2}{2m} \frac{\partial^2 \psi}{\partial x^2} + V(x)\psi
\]
Dans le code, on utilise des unités réduites : $m = \hbar = 1$.

\section*{Potentiel}
Le potentiel $V(x)$ est une barrière ou un puits rectangulaire défini par :
\[
V(x) =
\begin{cases}
V_0, & \text{si } x_0 \leq x \leq x_0 + a \\
0, & \text{ailleurs}
\end{cases}
\]

\section*{Paquet d’onde initial}
Le paquet d’onde initial est donné par :
\[
\psi(x, 0) = A \exp\left(i k x - \frac{(x - x_c)^2}{2\sigma^2} \right)
\]
où $A$ est un facteur de normalisation, et $k = \sqrt{2|E|}$ avec $E$ l'énergie de la particule.

\section*{Méthode numérique}
\subsection*{Schéma temporel}
Un schéma à pas alternés est utilisé pour intégrer l'équation de Schrödinger.

\subsection*{Conditions aux limites}
Des conditions aux limites de type Dirichlet sont implicites (bords à 0).

\subsection*{Animation}
Une animation est générée à partir de l'évolution de la densité $|\psi(x,t)|^2$.

\section*{États stationnaires}
On résout l'équation de Schrödinger indépendante du temps par diagonalisation :
\[
H \psi_n = E_n \psi_n
\]
Les fonctions propres sont normalisées et représentées avec leurs énergies.

\end{document}
